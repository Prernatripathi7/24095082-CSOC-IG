\documentclass[12pt]{article}
\usepackage{amsmath}
\usepackage{graphicx}
\usepackage{caption}
\usepackage{float}
\usepackage{titlesec}
\usepackage[margin=1in]{geometry}
\usepackage{setspace}
\usepackage{lipsum}

\titleformat{\section}{\large\bfseries}{\thesection.}{0.5em}{}
\titleformat{\subsection}{\normalsize\bfseries}{\thesubsection.}{0.5em}{}

\title{Multiple Variable Linear Regression}
\author{Prerna Tripathi}
\date{\today}

\begin{document}

\maketitle
\onehalfspacing

\section{Initial Procedures}
\subsection{Approach}
The common process across all the three parts of the assignment is to preprocess the data by removing all the missing values in the dataset which can lead to improper training of the model and also convert the feature 'ocean proximity'into a form the algorithm can understand and process.
The dataset is then divided into a 80:20 ratio into training and validation sets. The features are then also scaled using Z-score normalisation across all the three parts. Feature scaling is highly necessary for implementing gradient descent.
Few things as in the formation of the feature 'ocean proximity' was done with the help of gpt.
\section{Pure Python Implementation}
The pure python implementation implements multiple variable linear regression using gradient descent without using any libraries,using only loops and lists.The maximum no. of iterations is 500 and the learning rate is 0.01. The graph of convergence vs iteration shows that our gradient descent has converged well enough to a point of global minimum
Use of AI was done in areas using scikit-library to calculate regression metrics and in splitting the data.Otheriwse the implementation of the main gradient descent algorithm was done avoiding the use of AI.
\subsection{Analysis}
Convergence time: 356.3851 seconds
MAE= Training:50623.3644, Validation:50519.6950
RMSE= Training:69824.8406, Validation:69375.6927
R2-score= Training:69824.8406, Validation:0.6405
\begin{figure}[h]
    \centering
    \includegraphics[width=0.8\textwidth]{convergence_vs_iteration.png}
    \caption{Graph of Convergence vs Iteration:Pure Python}
    \label{fig:Convergence vs Iteration}
\end{figure}
\section{Numpy Implementation}
In this approach, the algorithm remains the same as the first part including the initial parameters, but instead of loops and lists Numpy library is used which through vectorized algebra and parallel operations improves the execution speed and shortens the code also.
Here also the learning rate is 0.01 and the no. of iterations is 500and the graph of convergence vs iterations shows that our gradient decsent has converged well enough to a minimum.
Use of AI was again restricted to the use of scikit-library functions and also to a few numpy operations which have been used in the code.
\subsection{Analysis}
Convergence time: 0.1059 seconds
MAE= Training:50480.7466,,Validation:50390.6575
RMSE= Training:69745.5760, Validation:69279.9713
R2-score= Training:69279.9713, Validation:0.6415
\begin{figure}[h]
    \centering
    \includegraphics[width=0.8\textwidth]{convergence_vs_iteration1.png}
    \caption{Graph of Convergence vs Iteration: Numpy Optimised}
    \label{fig:Convergence vs Iteration}
\end{figure}
\section{Scikit-learn Implementation}
\subsection{Approach}
The scikit-learn implementation uses the \texttt{LinearRegression} class, which abstracts away the algorithmic details. It provides built-in methods for model fitting, prediction, and performance evaluation.
Again the use of AI has been done to incorporate the scikit's operations.

\subsection{Analysis}
Fitting Duration:0.0132 seconds
MAE= Training:49752.9160, Validation:49964.0128
RMSE= Training:68758.3663, Validation:68201.5212
R2-score= Training: 0.6448, Validation=0.6526
\begin{figure}[h]
    \centering
    \includegraphics[width=0.8\textwidth]{Comparison.png}
    \caption{Comparsion Of MAE and RMSE over the three parts}
    \label{fig:Regression Metrics MAE and RMSE}
\end{figure}
\begin{figure}[h]
    \centering
    \includegraphics[width=0.8\textwidth]{Comparison1.png}
    \caption{Comparsion Of R2 score over the three parts}
    \label{fig:Regression Metrics R2}
\end{figure}

\end{document}
